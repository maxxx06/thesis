
\documentclass[../main.tex]{subfiles}

\begin{document}
\chapter{Conclusion et perspectives \maxime{NE PAS LIRE}}
%\addcontentsline{toc}{chapter}{Conclusion}
\label{perspective}
%\doublespacing %% For correction

\section*{Discussion}
\addcontentsline{toc}{section}{Discussion}

\textbf{Discussion partie informatique}\\
%explicabilité, import pour les biologistes, nous, informaticiens, sommes là pour ça.
Dans le monde de la biologie, l'identification des mécanismes sous-jacents expliquant une observation est un défi majeur. D'après \citep{lipton2017mythos}, les modèles dit "boîtes-noires" permettent difficilement une interprétation du résultat du modèle, et ainsi ne correspondent pas à la demande des biologistes. A l'opposé, nous retrouvons des modèles "boites blanches" qui, en plus d'identifier des mécanismes responsables de l'observable et d'obtenir une bonne prédiction, permettent d'émettre des hypothèses testables en laboratoire. De ce fait, nous avons développé deux modèles informatiques répondant aux doubles enjeux des biologistes : bonnes capacités prédictives et pouvoir d'explicabilité. En reprenant la terminologie de \citep{lipton2017mythos}, nos modèles assurent donc la \textit{causalité}, \textit{i.e.} l'explicabilité. Parmi les autres critères de \citep{lipton2017mythos}, nos modèles répondent également à la \textit{transparence}, c'est-à-dire, qu'ils peuvent être auditables par les biologistes. Ce critère traite la question de comment le modèle fonctionne, au niveau de l'algorithme, des paramètres  et du modèle entier. \label{contraintes-model}\\


%L'explicabilité permet de générer des hypothèses testables
Le modèle numérique présenté dans le chapitre \ref{tango} modélise un écosystème microbien étudié expérimentalement pour l'élaboration d'un fromage industriel. Pour rappel, notre stratégie itérative consistait à calibrer chaque souche métabolique pour inférer des potentiels d'interaction en communauté. De part cette stratégie, nous avons montré que la prédiction de notre modèle numérique était de bonne qualité. Nos résultats de simulation, à l'échelle de chaque souche métabolique et en communauté, sont numériquement proches des données expérimentales. En raison de cette précision, de la généralité du modèle et de l'utilisation explicite de mécanismes biologiques, nous appelons ce modèle numérique un modèle numérique mécanistique. De plus, la \textit{causalité} du modèle est assurée par l'identification de voies métaboliques activées, de flux de consommation et ou de production ainsi que par la contribution de chaque modèle métabolique. Concernant la souche de \plantarum, nous avons observé pour la première fois en lait l'activation de la voie hétérolactique, représenté par un flux non-nul de la voie de la transketolase. De plus,nous avons observé une utilisation du citrate par \plantarum et une préférence d'utilisation du lactose au regard du lactate par \freud. Notre modèle numérique explique ces résultats par une valeur de xflux de consommation de ces composés. Nous avons également implémenté un partage équilibré des ressources, lorsqu'elles deviennent limitantes, entre les organismes, affinant la prédiction des concentrations des métabolites. Ce mécanisme a permis de mettre en avant les interactions bactériennes révélées dans le chapitre \ref{tango}. Toutes ces explications fournies par le modèle permettent ainsi de générer des hypothèses testables expérimentalement, qualifiant ainsi notre modèle numérique de modèle explicable. Enfin, nous assurons la \textit{transparence} en explicitant les noms des métabolites, des réactions et des voies métaboliques. \\


%Le modèle numérique Tango modélise un écosystème microbien étudié expérimentalement pour l'élaboration du fromage industriel. \\
%La prédiction de ce modèle est de bonne qualité car les résultats sont numériquement proches des données expérimentales \\
%Nous assurons la causalité avec l'activation de voies métaboliques, flux entrant, sortant, contribution chaque souche. [Puis, pour chaque item les détailler.]

%Dans le cadre du jumeau numérique, nous avons implémenté un partage équilibré des ressources entre les membres de la communauté lorsque cette dernière devient limitante (quel critère de succès ? (prediction,causalité,transparence).
%Comme vu dans le chapitre correspondant, cette limitation à pour but d'éviter des valeurs négatives et a permis de déduire des phénomènes d'interactions au sein du consortium bactérien. Au niveau de la communauté, notre modèle explique les concentrations des métabolites suivis en métabolomique en montrant la contribution relative de chaque espèce dans la production ou la consommation de ces composés.


% Créer des modèles explicables pour les biologistes ne permet pas seulement d'expliquer le lien entre les données d'entrée et la sortie, mais également, de générer des hypothèses testables en laboratoire)\maxime{pas utile, introduire par paragraphe ce qu'était chaque modèle en une phrase.}.
%En effet, au sein du modèle numérique de la fermentation du fromage, de nouveaux flux entrants, sortants et nouvelles activations de voies métaboliques peuvent expliquer les observations expérimentales et ainsi être vérifier en laboratoire. A l'échelle de la souche de \plantarum, nous avons mis en avant une activation de la voie hétérolactique en lait, représenté par des flux non nul au sein de la voie métabolique de la transketolase. Ou encore, chez la souche de \freud, une préférence pour consommer le lactose à la place du lactate représenté par une valeur de flux de consommation pour le lactose plus importante. Au niveau de la communauté, notre modèle numérique permet de prédire la contribution de chaque souche en pour la consommation ou la production des composés. Ainsi, l'impact et le rôle d'une souche sur cet écosystème du fromage peut-être testé \textit{in silico} et vérifié expérimentalement. \maxime{rajouter le côté précision numérique (bonne prédiction) et transparence. Elles sont assurées par blabla. La causalité par ce modèle est assurée par ........}

%Chacune de ces items est une hypothèse potentiellement testable. \\
%Comment nous assurons la transparence ? Pour assurer la transparence, on dit les noms de métabolites etc etc.

%Les jumeaux numérique boites blanches diffèrent des boites noirs par l'utilisation explicite des mécanisme bio ainsi que modèle logique
% Nos deux approches, numérique boîte-blanche ou par raisonnement, diffèrent des modèles boîtes noires en répondant au critère de \textit{causalité}, inférant explicitement des mécanismes biologiques, en les formalisant sous forme de contraintes numérique ou de logique pour révéler de nouveaux phénomènes.
% Dans le cadre du jumeau numérique, nous avons implémenté un partage équilibré des ressources entre les membres de la communauté lorsque cette dernière devient limitante. Comme vu dans le chapitre correspondant, cette limitation à pour but d'éviter des valeurs négatives et a permis de déduire des phénomènes d'interactions au sein du consortium bactérien. Concernant l'approche par raisonnement, nous avons formalisé sous forme logique la définition biologique d'un substrat limitant, d'un composé échangeable dans une communauté et révélé un potentiel de coopération et de compétition en se basant sur ces définitions.\\


Le modèle discret du chapitre \ref{ccmc} modélise l'interaction métabolique entre organismes au sein d'un écosystème microbien et prédit le potentiel de compétition et de coopération entre ces organismes. Avec l'approche par raisonnement, nous avons formalisé des définitions biologiques sous forme de règles et contraintes logiques: à partir d'un ensemble nutritif nous calculons le potentiel métabolique pour chaque souche et en communauté; \textit{scope}; les échanges métaboliques; \textit{exchanged}; les substrats en compétition; \textit{polyopsonistic}. Cette formalisation explicite nous permet de calculer des potentiels d'interaction, compétition et coopération, et garantit une bonne confiance dans les prédictions de notre modèle.
D'une part, nos prédiction permettent d'obtenir des résultats similaires ceux fournis les outils numériques existants qui font référence, SMETANA et MICOM, pour les mêmes données.
D'autre part, notre logiciel réussit une batterie de tests induits par un benchmark très complet, que nous avons conçu pour valider les prédictions et scores spécifiques d'un écosystème, évaluer le passage à l'échelle et vérifier le criblage haut débit des communautés.

Tout comme le modèle numérique mécanistique, ce modèle discret assure également la \textit{causalité}: le potentiel de coopération est expliqué par le nombre de métabolites échangés ainsi que par les espèces intervenant dans l'échange; tandis que le potentiel de compétition est expliqué par le nombre de consommateurs associés à un substrat limitant. Ainsi, le proxy de la coopération, représenté par la notion de métabolites échangeables, est expliqué en termes biologiques ainsi: parmi deux organismes A et B, un métabolite est échangeable si, il est productible par l'organisme A et consommable par l'organisme B uniquement lors d'un échange.
De même, la compétition est expliquée en termes biologiques: un substrat nutritionnel est en compétition si ce dernier est limitant et co-consommé. Toutes ces explications issues de notre modèle permettent de générer des hypothèses testables en laboratoire.
La production d'un métabolite échangé entre organismes coopérants A et B peut être détecté en métabolomique par LCMS ou par phénotypage biochimique ciblé, et associé avec la présence ou l'absence de l'espèce A.
La consommation d'un métabolite par deux organismes en compétition peut être testé de façon similaire.

Enfin, notre modèle assure également la \textit{transparence} car chaque règle logique du modèle est lisible, vérifiable et critiquable par les biologistes. En reprenant la notion biologique de coopération plus haut, un métabolite est dit échangé s'il est dans le \textit{scope} d'un producteur, qu'il soit reactant d'un consommateur et qu'il ne soit pas dans le \textit{scope} du consommateur.
%Le modèle discret modélise l'interaction métabolique entre organismes microbiens dans un écosystème microbien et prédit le potentiel de compétition et la coopération entre ces organismes. \\
%Les prédictions de ce modèle sont bonnes pour deux raisons: la première est que nos résultats sont au moins aussi bons que des outils numériquement plus coûteux. d'autre part, à l'aide d'un benchmark répondant à nos attentes. \\

%Comment nous assurons la causalité ? le nombre de métabolites échangés, le nom des espèces intervenant, le nombre d'espèces intervenant.  donner un exemple pour la compétition: ça veut dire qu'il existe, pour un métabolite limitant, un nombre de consommateurs supérieur à 1. donner l'exemple pour la coopération également. \\
%Comment nous assurons la transparence ? Tout cela est dû au fait que chaque inférence suit une règle biologique, et lisible et critiquable par les biologistes.\\

%D'autre part, notre modèle par méthode de raisonnement, en plus de pouvoir passer à l'échelle, est intrinsèquement lié à la famille de modèles interprétables. En effet, grâce à la méthode de résolution de Clingo, l'ensemble de réponse qui en résulte satisfait les contraintes et règles logiques écrites et ainsi, éviter des possibles biais dans la résolution. Ce pouvoir de modélisation explicable par nature permet de soulever des hypothèses testables sans processus d'inférence numérique coûteuse. En effet, le nombre de métabolites échangés ainsi les espèces productrices et consommatrices intervenant dans ces échanges, la plus value d'ajouter ou retirer une espèce, génèrent des hypothèses testables rapidement.\\




%À partir des besoins des bio, je peux les aiguiller sur la nature de la modélisation. Développer
% Ces deux approches ont montré qu'ils sont plus ou moins appropriés aux besoins des biologistes. \maxime{ne pas hésiter à illustrer les questions que l'on se pose -> quelles voies activées : numérique etc etc}. \maxime{C'est le moment de mettre mon tableau. }
Les modèles explicables développés dans cette thèse ont été élaborés pour répondre avec précision à des enjeux biologiques précis. Dans les deux cas, identifier la bonne méthode correspondante à ses enjeux n'a pas été trivial mais la conclusion d'une analyse menée en concertation avec les collègues biologistes.
Le critère que les modèles doivent être prédictifs et explicables ne permet pas en soi de choisir entre un modèle numérique et discret.

% Enjeux que l'approche discrète répond: passage à l'échelle, hypothèse de compétition/coopération, explication mécanistique, ordonnancement d'évènement (temporalité), sélection de consortium bactérien, comparaison de communauté, scope

L'approche discrète développée répond à de multiples enjeux que sont: la résolution de problèmes combinatoires permettant de passer à l'échelle de communautés naturelles, d'émettre des hypothèses sur de potentielles interactions bactériennes, d'apporter une explication structurelle des phénomènes observés, d'obtenir des \textit{scope} de métabolites productibles et consommées par un plusieurs organismes et d'obtenir une précision des résultats dû à l'inférence des règles biologiques. Chacun des modèles mécanistiques que nous avons développés permet de répondre à ces enjeux plus ou moins efficacement. Le modèle numérique a démontré sa polyvalence pour répondre à ses enjeux et est particulièrement adapté pour donner une dimension numérique de l'explicabilité, de scope métaboliques et des hypothèses testables. Lee modèle par raisonnement est quant à lui pertinent pour sa capacité à déterminer des potentiels d'interaction pour des grandes communautés en résolvant des problèmes combinatoires et de passer ainsi à l'échelle. De plus, ce modèle procure une explication détaillée et rapide du fonctionne de la commauté: par exemple, l'ensemble des métabolites échangés ainsi que les ensembles de producteurs et de consommateurs intervenant pour l'échange de ce métabolite. En somme, chaque enjeu peut être résolu par un modèle, et donc, un unique modèle hybride n'est pas forcément nécessaire.


% En revanche, le degré de précision le permet. Par exemple, les deux modèles permettent de montrer des voies métaboliques activables, mais seul le modèle numérique permet d'identifier celles qui sont réellement activées ou non à un temps donné. Le modèle discret permet de définir des ensembles de métabolites productibles tandis que le modèle numérique précise en quelle quantité ils sont produits. Enfin, le modèle numérique est adaptable sur des écosystèmes contrôlés alors que le modèle discret est généralisable sur des écosystèmes naturelles composés de plusieurs centaines de bactéries. Ainsi, le modèle numérique avait pour enjeu de créer un modèle de fermentation du fromage en reproduisant les concentrations observées expérimentalement. Le modèle discret était focalisé sur la comparaison au débit de potentiels de coopération et de compétition d'écosystèmes microbiens. Ce travail a démontré qu'à élaborer un modèle hybride pour la modélisation l'écosystème microbien n'est pas nécessaire, mais que la difficulté était de choisir le modèle le plus efficace pour répondre à l'enjeu biologique.
%Ces modèles, logiques et numériques, ont montré leur capacité à expliquer les mécanismes cachés à partir d'un enjeu biologique. En revanche, choisir la bonne méthode pour répondre à ses enjeux n'est pas trivial. Nous avons pour cela dû formaliser des enjeux à partir de discussions entre experts. Nous pouvons classer ces enjeux en deux familles: la précision des résultats et la flexibilité du modèle. A partir de ces deux classes nous pouvons recommander l'utilisation d'un modèle numérique ou d'un modèle logique en prenant en compte les possibles limites de ces modèles par rapport aux enjeux biologiques. Ainsi, pour une reproduction fidèle des données expérimentales, un modèle numérique comme celui du chapitre \ref{tango} conviendrait mieux. En revanche, dans un but de construction de consortium bactérien, de comparaison de communauté bactérienne ou encore d'ordonnancer des processus de manière haut débit, un modèle discret est plus adapté.

Notre étude permet cependant d'évaluer l'opportunité que représente la construction des modèles hybrides pour des écosystèmes microbiens complexes.
Quelles auraient été les propriétés d'un modèle hybride, combinant les avantages de la précision numérique d'un jumeau numérique avec celles du passage à l'échelle de communautés naturelles d'un modèle discret.
Dans un soucis de satisfaire les contraintes explicitées dans le paragraphe \ref{contraintes-model}, ce nouveau modèle hybride doit toujours être explicable, en donnant de bonnes prédictions numériquement proches des valeurs expérimentales et qu'il puisse identifier des potentiels d'interaction à grande échelle dans le but de générer des hypothèses testables sur les voies métaboliques activées, les espèces impliquées dans un échange métabolique ou encore celles en réelle compétition.




% Dans un souci de satisfaire les biologistes, ce modèle hybride doit toujours être explicable, c'est- à dire, promouvoir de bonnes prédictions, et générer des hypothèses testables.
Deux pistes peuvent être explorées : améliorer le passage à l'échelle du modèle numérique ou enrichir le modèle discret. Nous avons opté pour le second choix en proposant un premier enrichissement temporel. En utilisant la logique temporel \citep{Cabalar2019}, il est possible, à l'aide de règles logiques temporelles, d'identifier quand un métabolite est productible. Cet ajout temporel permettrait d'affiner la prédiction du potentiel de compétition: deux espèces sont en compétition si pour un même substrat limitant, au moins deux espèces le consomment \textsl{au même moment}. Nous avons par la suite permis la sélection de communauté pour la construction d'un design expérimental. Cet ajout permet d'améliorer l'explicabilité du modèle puisque chacun des modèles proposés par le modèle discret est expliqué par les critères de sélection en amont.

Le modèle numérique représente le but final à atteindre de part sa précision numérique. Pour améliorer son passage à l'échelle, les paramètres inférés pour chaque souche métabolique peuvent être déduits à partir d'un méta-modèle pré-entraîné \textbf{(ref)}. Dans un second temps, le flux de consommation et de production peut être calculés à partir de la concentration des métabolites d'intérêt à chaque point de temps. Enfin, nous pouvons discrétiser le temps et calculer un FBA dans un intervalle de temps plus important, réduisant ainsi le temps de calcul mais augmentant l'approximation entre deux pas de temps.



% Cette étude à montrer que nous n'avons pas eu besoin de faire un modèle hybride. mais qu'elle aurait été les propriétés du grand modèle hybride en combinant les avantages de ccmc et tango? [Il faut reprendre les étapes : en quoi il ferait de bonnes prédictions, etc ON VEUT TOUJOURS QU'IL SOIT EXPLICABLE : \emph{causalité} et \emph{transparence}.]
% Deux pistes: soit du numérique vers le passage à l'échelle soit du discret vers un enrichissement.
% Pourquoi enrichir le discret est mieux ?
% Type d'enrichissement: un métabolite est compétitif si deux espèces consomment en même temps ce métabolite sans la nécessité de parler de concentrations ; un métabolite est témoin de la coopération si ...

% Travaux futurs: à quoi ressemble le tango en mode hybride ? Est ce que ca serait toujours générateur testable ?
%\maxime{uniformiser les mots : numérique ? jumeau? discret}
\maxime{expliciter pourquoi c'est dur LCMS, car c'est la résultant production-consommation}
%\maxime{page 5 reformuler scope/iscope}
\maxime{'prise en compte de la temporelle' dernier paragraphe}

\textbf{Discussion partie biologique}\\
%Comment la modélisation par topologique et en flux peut aider le biologiste à identifier les interactions bactériennes favorables et défavorables au fonctionnement de l’écosystème ?

La fermentation bactérienne est au c\oe{}ur de la fabrication fromagère et elle se caractérise par une acidification du milieu, une production de composés d'arômes et enfin par une croissance bactérienne. Les échanges de métabolites entre espèces favorisent le fonctionnement de l'écosystème.

%Comme nous avons pu le voir dans cette thèse, la modélisation discrète et en flux ont permis de générer des hypothèses testables sur les interactions bactériennes favorables et défavorables au fonctionnement de l'écosystème. Dans le chapitre \ref{tango}, la fermentation bactérienne était le c\oe{}ur du sujet se caractérisant par une acidification du milieu, une production de composés d'arômes et enfin par une croissance bactérienne pour la production de fromage. Dans le cas de la modélisation discrète du chapitre \ref{ccmc}, nous nous sommes plus intéressé aux potentiels métaboliques, i.e. de production, de consommation, de coopération, de compétition etc. \\

L'approche discrète du chapitre \ref{ccmc} identifie des hypothèses biologiques en comparant des ensembles, de métabolites produits et consommés, par entité biologique au sein d'un écosystème. Parmi ces ensembles, nous avons calculé les ensembles de métabolites: (i) échangeables entre différentes souches, (ii) déjà présents dans le milieu extra-cellulaire, (iii) nouvellement disponibles libérés au cours d'une fermentation bactérienne par exemple, (iv) pouvant être limitant. L'analyse comparée de ces ensembles a permis de prédire les espèces en compétition, les échanges et les souches impliquées dans les échanges. Ces échanges peuvent impacter la croissance des bactéries, la production de composés d'intérêts et le pH. En plus de l'identification d'interactions en comparant des ensembles, nous avons calculé des potentiels de coopération et de compétition, ce qui ajoute une dimension pseudo-quantitative à l'interaction et permet de comparer des communautés de divers écosystèmes entre eux.

La dimension quantitative d'une interaction bactérienne est apportée par le modèle numérique mécanistique du chapitre \ref{tango}. Cette dimension est caractérisée par la prédiction de la concentration des métabolites, et la quantité de biomasse produite, correspondant au taux de croissance de l'espèce. Les interactions sont identifiées en analysant les flux de métabolites (i) entrant dans la cellule bactérienne et (ii) sortant vers le milieu extracellulaire. Par exemple, à l'échelle de chaque cellule, tous les modèles métaboliques ont un flux de consommation du lactose non nul, suggérant une compétition pour ce dernier. Ou encore, le lactate est produit par les bactéries lactiques et consommé par \freud symbolisé respectivement par un flux de lactate sortant et entrant. Ces mécanismes révèlent une coopération entre les bactéries lactiques, \plantarum et \lactis, et la bactérie propionique \freud. Cette analyse est faite grâce aux méthodes d'analyse par  contrainte des flux (FBA, FVA) \citep{Orth2010, Mahadevan2003}sous l'hypothèse de  maximisation de la biomasse. Il est possible de faire varier cette fonction objective, par exemple maximisation de la synthèse d'ATP ou de lactate, la distribution des flux changerait et donc, de nouvelles hypothèses d'interactions verraient le jour.

Ces interactions ont des impacts cinétiques sur la concentration des composés d'arômes et sur la densité bactérienne. Au cours du temps, la concentration et la densité bactérienne varient, et une croissance nulle ou ralentie d'une souche peut-être expliquée par une compétition pour un substrat ou par un précurseur limitant, au sens biologique. La production d'un composé d'arôme peut également être ralentie ou annulée, suggérant également une compétition. Ces hypothèses ont été mises en avant par une analyse dynamique de l'équilibre des flux (DFBA) \citep{Mahadevan2002} et a permis de proposer des valeurs de contribution relative de chaque espèce à la consommation et la production de composés d'intérêts. Nous avons ainsi suggéré une indication des producteurs et des consommateurs pour chaque composé pouvant être testés en laboratoire. La validation biologique pourrait être réalisée par un knockout des gènes correspondants. En plus de ces impacts, nous avons pu déduire quelles sont les espèces participant le plus à l'acidification du milieu, au regard de notre modèle de pH. \\



%Quels aller-retours peut-on envisager entre modélisation et expérimentation biologique pour affiner cette compréhension ? en gros … Si on devait refaire le projet Tango toi et moi en partant de zéro, comment arriver à un résultat d’identification d’interaction au moindre effort ? peut-on introduire plus tôt la modélisation ? Est ce que la metaT était indispensable ? …
Les deux approches, discrètes et numériques que nous avons développées dans cette thèse, ont permis de générer des hypothèses testables, affinant la compréhension des interactions qui favoriserait la croissance bactérienne et/ou la production de métabolites. A partir de ces hypothèses, nous discuterons des expérimentations biologiques permettant de valider ou d'invalider les hypothèses afin d'améliorer les modèles explicatifs. Nous traiterons uniquement le cas où des génomes annotés et les souches correspondantes sont à notre disposition. \\


Les approches numériques et discrètes que nous avons développées ont permis de détecter des interactions bactériennes. Dans le cas d'un échange de métabolites entre deux souches, nous avons pour les deux approches la bactérie productrice et consommatrice ainsi que le sens de l'échange. Par exemple, le lactate, mis en avant par notre implémentation du dFBA, est produit par les bactéries lactiques et consommés par \freud, la phénylalanine, révélée par SMETANA \citep{Zelezniak2015} et MiCOM \citep{diener2020}, produite par \plantarum et consommée par \freud. Ces deux approches approches permettent également de nous indiquer des souches co-consommatrice de substrats qui peuvent conduire à une compétitionentre souches si c'est substrats devenaient limitants. Par exemple, le lactose, co-consommé par les consortium, ou encore la glycine, produite par \plantarum et co-consommée par \lactis et \freud. Toutes ces interactions révélées \textit{in silico} avec des outils \textit{a priori} ou \textit{sans a priori} peuvent être validées expérimentalement au moyen d'une comparaison entre des co-cultures et des cultures pures. Afin de vérifier l'impact de ces interactions, la comparaison se porterait sur la croissance des bactéries ainsi que sur le dosage de métabolites.

Dans le cas d'une hypothèse de co-consommation d'un substrat, une décroissance bactérienne et/ou une diminution de la concentration de ce métabolite devraient être observées en co-culture. Afin de vérifier les souches impliquées, un marquage isotopique à froid de ce métabolite suivi d'une spectroscopie avec la méthode Raman \citep{Wang2020,CUI2022100187} permettrait d'identifier les consommateurs de ce métabolite. 
% Dans le cas d'une coopération syntrophique, i.e. où nous avons un échange d'une productrice vers un consommateur, il existe des sous-cas pouvant expliquer l'observation en co-culture. Le premier point attendu est une \textit{sous production} de ce métabolite échangé. Il existe, à notre connaissance, 3 hypothèses biologiques pouvant expliquer cette observation, pouvant être: (1) une inhibition de la croissance de la bactérie productrice empêchant la production de ce dernier, un échange métabolique et/ou une répression transcriptionnelle ou traductionnelle de la voie de synthèse de ce métabolite par l'espèce productrice. Le second point se traduirait par une \textit{sur production} de ce métabolite. Il existe également 3 hypothèses biologiques pouvant expliquer cette observation : croissance bactérienne favorisée, activation du processus de production et/ou échange d'intermédiaires métaboliques.  
 Les activations et les inhibitions de productions métaboliques peuvent être mises en avant avec les données méta-transcriptomique ou métaprotéomiques. Pour révéler expérimentalement ces échanges d'intermédiaires métaboliques (1), il faut valider indépendamment la production de ce métabolite par l'espèce productrice en culture pure et la consommation par l'espèce consommatrice avec un marquage isotopique.

%Comme indiqué plus haut, la méthode de jumeau numérique a permis de mettre en évidence des interactions en comparant les flux d'entrée et de sortie des modèles métaboliques avec des méthodes de flux statique. Cette comparaison met en évidence les métabolites échangés, les espèces qui relarguent le métabolite et les espèces qui le consomment. En mettant en place une comparaison des cultures pures de chaque espèce avec les espèces en co-culture, du point de vue des dosages  biochimiques, nous pouvons identifier les acteurs de l'interaction.
%Pour vérifier qu'elles sont les consommateurs, le métabolite échangé peut-être suivi avec un marquage isotopique. Lors d'un dFBA avec communauté, nous avons en plus le dynamisme du métabolisme de communauté au cours du temps. Dans un premier temps, les flux de production et de consommation de chaque espèce pour chaque métabolite ainsi que la concentration de ce dernier. Pour vérifier les interactions mis en évidence par cette donnée, l'usage de données metatranscriptomiques ou de métaprotéomique peut servir à les valider ou à les réfuter, en regardant l'expression des gènes associés au réactions.
%
%En comparant notre approche \textit{a priori} avec des méthodes \textit{sans a priori}, SMETANA et MiCOM, nous avons évalué les potentiels interactions que notre communauté fromagère peut développer. Nous avons pour chacun des outils des ensembles de métabolites échangés ainsi que les orientations des échanges. Comme précédemment, une comparaison pure/co-culture peut-être faite afin d'identifier les productions individuelles et utiliser un marquage isotopique afin d'identifier les consommateurs. Enfin, une validation avec des données métaTranscriptomique et métaprotéomique peut être faite. \\

%Contrairement à notre méthode de jumeau numérique, notre modèle discret a mis en avant les interactions en se basant sur des ensembles: métabolites échangés (1), souches qui participe à ses échanges ainsi que le sens (2) ainsi que des souches qui consomme le même substrat appartenant au milieu de culture et à l'ensemble des métabolites échangés (3).  En utilisant une comparaison de pure culture et de co-culture des croissances bactériennes et d'un dosage métabolique pour chaque espèce en cinétique , les points (2) et (3) peuvent être vérifiés. Dans le cas d'une hypothèse de co-consommation (3), une décroissance bactérienne ou/et une diminution de la concentration de ce métabolite serait observé. Dans l'identification des souches (2),  deux cas peuvent avoir lieu. Dans le premier cas, une \textit{sous-production} de ce métabolite échangé en co-culture est observée. Les hypothèses biologiques sous-jacentes expliquant cette observation sont les suivantes: une inhibition de la croissance bactérienne empêchant la production de ce métabolite, un échange métabolique ainsi que le sens de l'échange et enfin, une inhibition de la production de ce métabolite par l'espèce productrice. Dans le second cas, une \textit{sur-production} de ce métabolite est observé en co-culture. Les hypothèses biologiques expliquant cette observation sont les suivantes: croissance bactérienne favorisée, activation du processus de production et un échange d'un intermédiaire métabolique. Les activations et les inhibitions de productions métaboliques peuvent être mis en avant avec les données méta-transcriptomique ou métaprotéomiques. Pour révéler expérimentalement ces échanges d'intermédiaires métaboliques (1), il faut valider indépendamment la production de ce métabolite par l'espèce productrice et la consommation par l'espèce consommatrice. \\

\textbf{Retour d'expérience de la thèse}\\
Au cours de la collaboration établie avec les biologistes dans cette thèse, nous avons pu mettre en évidence deux façons d'exploiter les modèles développés. Dans un premier temps, seul le \textit{résultat} du modèle est important afin de vérifier expérimentalement les prédictions. Par exemple, au sein du chapitre \ref{tango}, nous avons mis en avant de nouvelles interactions métaboliques potentielles, à la fois avec notre modèle  avec \textit{a priori} et les outils \textit{sans a priori}, pouvant être validées ou réfutées grâce aux moyens que nous avons développés plus haut. Dans un second temps, pour une réexploitation des résultats de la thèse, la propriété de \textit{généricité} des outils et du savoir-faire est importante. Cette généricité permet l'utilisation pour une autre question de recherche ou avec des données d'entrées différentes. Par exemple, les outils développés des chapitres \ref{ccmc} et \ref{enrichissement} ont permis l'utilisation de réseaux métaboliques appartenant à n'importe quels écosystèmes, et permettent de s'intéresser à la compétition, coopération, sélection de communautés selon différents critères biologiques etc. Également, le degré de \textit{diffusion} des logiciels développés est un facteur à prendre en compte pour un travail interdisciplinaire. En effet, le déploiement de nos outils sur une plateforme, telle que Galaxy \citep{10.1093/nar/gkac247} par exemple, permettrait de garantir un accès simplifié pour les biologistes et d'approfondir leur recherche sur la compréhension des interactions bactériennes.


\section*{Bilan des contributions}
\addcontentsline{toc}{section}{Bilan des contributions}

Au sein des travaux de cette thèse nous avons développé plusieurs méthodes d'analyse de communautés microbiennes applicables à des communautés appartenant à un écosystème simplifié, ou bien, à des communautés simulées provenant d'écosystèmes naturels complexes, en proposant dans un premier temps de caractériser la coopération et la compétition au sein d'une communauté et dans un second temps, de proposer un modèle de sélection de communauté pour permettre l'analyse numérique de communautés intéressantes.\\
Notre méthodologie numérique se base sur la connaissance \textit{a priori} du système biologique d'étude ainsi que sur la grande disponibilité des données. Ainsi, notre approche itérative devrait être applicable pour toute communauté où ces contraintes sont respectées. En effet, le raffinement de chaque modèle est un processus universel consistant à ajouter, éteindre ou modifier des réactions. En se basant sur l'analyse de la variance des flux et de la connaissance avec \textit{a priori}, les chemins métaboliques utilisés ont pu être déterminés et la cohérence de ces résultats vis à vis de la littérature a été vérifiées. De plus, l'ajustement des mécanismes biologiques, tels que la consommation du lactose ou du lactate, à travers l'optimisation des paramètres, a été déduit en observant les courbes de simulations dynamiques. Enfin, nous avons montré qu'il était possible, à partir d'un mécanisme équitable de partage des ressources et d'un "tunning" à l'échelle de chaque bactérie, de mettre en avant des interactions bactériennes en communauté et d'obtenir des prédictions numériques satisfaisantes. Ces résultats ont fait l'objet d'un papier soumis dans metabolic engineering.\\

Nous avons par la suite cherché à définir les limites de l'analyse des interactions microbiennes dans le cas le plus pauvre, c'est-à-dire, avec uniquement des données génomiques. En basant notre méthodologie développée sur le principe de raisonnement et construit comme une extension du logiciel MISCOTO \citep{Frioux2018}, les propriétés  d'explicabilité des modèles et le passage à l'échelle de communautés de grande taille ont été conservés. Grâce à cette approche d'abstraction booléenne, nous avons caractérisés des potentiels de coopération et de compétition permettant une comparaison de communautés entières et non deux à-deux. Même si la méthode n'a pas la précision des approches numériques, elle compense par sa capacité de criblage au débit, permettant d'effectuer un premier filtre de caractérisation de l'écosystème microbien. Lors de la comparaison avec des méthodes quantitatives, nous avons démontré que la tendance de nos scores, issu d'un raisonnement booléen, était corrélée avec des méthodes basées sur des méthodes par contrainte. De part le faible temps d'exécution de notre approche, ces résultats confortent l'utilisation d'une méthode moins précise mais peu coûteuse pour permettre d'identifier manuellement des ensembles considérés pertinents du point de vue de la compétition, de la coopération, des substrats limitants ou bien des métabolites échangeables. Enfin, nous démontrons l'indépendance de notre approche vis-à -vis de la méthode de reconstruction des réseaux métaboliques ainsi que de l'écosystème d'origine des génomes. \\

Nous avons par la suite cherché à enrichir ces modèles logiques du point de vue de l'identification de communautés pertinentes en ajoutant des contraintes fournies par l'utilisateur ou l'utilisatrice. En effet, dans un monde idéal, un modèle hybride où le formalisme logique et numérique communiquent, permettrait à la fois une sélection de communautés pertinentes et une analyse numérique de ces dernières. La méthode de \citep{Frioux2018} est une première étape dans la construction d'un tel modèle qui sélectionne uniquement des communautés minimales, en minimisant le nombre d'échanges métaboliques au sein de la communauté. En s'inspirant de cette méthode, nous avons contruit un prototype qui, à partir d'un ensemble de réseaux métaboliques, l'utilisateur ou l'utilisatrice peut sélectionner une communauté en filtrant sa taille, sa composition taxonomique, le nombre de métabolites cibles devant être produits. Par ailleurs, l'ordre et le choix d'optimisation est également libre à l'utilisateur ou l'utilisatrice. Grâce à ce travail, la taille de l'ensemble de communautés trouvés, respectant les règles et les contraintes logiques définies en amont, est réduite et l'utilisation d'un modèle numérique précis et performant sur des communautés de petites tailles peut être ainsi utilisé. Ce travail, vers un modèle hybride, est l'objet d'un papier scientifique en cours de rédaction.

\newpage
\section*{Perspectives}
\addcontentsline{toc}{section}{Perspectives}

Avec ces trois chapitres, nous avons contribué à l'avancement des analyses des communautés microbiennes sur le plan numérique et discret. Cependant plusieurs points d'amélioration subsistent. A court terme, valoriser davantage l'intégration des données métatransciptomiques serait intéressant. En effet, dans le chapitre \ref{tango}, nous avons utilisé ces données hétérogènes pour valider des interactions mises en évidence par SMETANA en regardant l'expression des gènes associés à la production et à la consommation des métabolites présumés échangés. Il pourrait être intéressant de contraindre les réseaux métaboliques avec les données méta-transcriptomiques avec différentes méthodes \citep{Jenior2020,Agren2012,Becker2008} et de les comparer avec nos données simulées au niveau des voies métaboliques exprimées ainsi que par rapport aux données de métabolomiques. De plus, nous pourrions nous demander si seul les régulations des expressions géniques suffisent pour obtenir un modèle numérique précis, ce qui aurait pour conséquence de réduire le temps passé au raffinement et à la  calibration de modèles métaboliques. Ces données pourraient également servir de contrainte \textit{a priori} pour les modèles discrets du chapitre \ref{ccmc} et \ref{enrichissement}. En effet, inférer des règles logiques sur l'expression de voies métaboliques connues peut changer les potentiels d'interactions de communautés microbiennes. \\

À moyen terme, la construction d'un modèle hybride des écosystèmes microbiens pourrait être faite. Même si nous avons conclu qu'un modèle hybride n'est pas nécessaire, ce type d'approche pourrait apporter une plus-value. En effet, les métabolites suivis en dynamique dans le modèle numérique du chapitre \ref{tango} proviennent d'une question biologique. Coupler ce modèle numérique à une approche de criblage, comme le modèle logique du chapitre \ref{ccmc}, permettrait de fournir des métabolites d'interêts en entrée. Une difficulté mise en évidence, est le grand nombre de composés échangés détecté par une telle approche. Dans le but de filtrer ces solutions, un premier apport du modèle numérique serait de vérifier la faisabilité de production des composés identifiés par le modèle logique. Cette faisabilité pourrait-être transmise aux modèles logiques comme une donnée d'entrée sous forme de règles et ou contraintes. Nous aurions à terme, un modèle hybride couplé où le formalisme numérique et le formalisme discret communiquent permettant l'analyse de communautés microbiennes.\\

Enfin à plus long terme, la construction d'un modèle de jumeau numérique du processus de la fermentation bactérienne. Selon la définition de L'INRAE, un jumeau numérique est "défini comme une représentation numérique de l'object ou du système d'intérêt  (physique ou biologique ; e.g. cellule, tissu, plante, animal), incluant des variables décrivant son état et son évolution de façon dynamique" \footnote{\url{https://digitbio.hub.inrae.fr/thematiques/vers-le-jumeau-numerique\#Jumeaux}}. Nous avons des données d'entrée dynamique sur le processus de fermentation (métatranscriptomique, métabolomique et cinétique), un algorithme de modélisation et de simulation. Une caractérisation manquante est la boucle de rétroaction du modèle sur l'état du processus modélisé: \textit{i.e.} une prise de décision.



%Les analyses du chapitre \ref{tango} ont révélé des interactions biotiques permettant d'expliquer les observations biologiques. Les données méta-transcriptomiques ont été seulement utilisées pour valider ou réfuter les échanges métaboliques \textit{ad hoc}, et n'ont pas été utilisées pour contraindre les GEMs.  Le consortium bactérien analysé au sein de ce chapitre fait partie d'un projet industriel où l'impacte des facteurs abiotiques, comme la température, ont analysé. Une amélioration possible du modèle numérique serait de prendre en compte ces facteurs abiotiques dans le modèle dynamique. Toujours en se concentrant sur les réseaux métaboliques, \citep{Pettersen2023} ont effectué de l'inférence de paramètre de température sur les modèles métaboliques à l'échelle du génome. \\
%
%Dans le chapitre \ref{ccmc} et surtout le chapitre \ref{enrichissement}, nous nous sommes focalisés sur l'analyse à grande échelle de communautés bactériennes en apportant une notion pseudo quantitative, avec les scores de coopération et compétition, et en filtrant des communautés candidates. Dans l'optique d'apporter plus de précision dans chacun de ces modèles, nous pouvons utiliser la méthode d'analyse par équilibre des flux présenté dans le chapitre \ref{ch:edla} pour contraintes les ensembles de solutions. Il existe clingoLP \citep{Janhunen2017}, une extension du solveur clingo pour la résolution de contraintes linéaires. En effet, dans \citep{Thuillier2022}, ils utilisent un solver intégrant des contraintes linéaires dans de l'ASP pour modéliser l'hypothèse que le système est à l'équilibre. Ainsi, les proxy de la coopération et de la compétition, que sont respectivement les métabolites échangeables et les substrats limitants, identifiés grâce aux contraintes logiques seront restreint avec les contraintes linéaires. Un deuxième point d'amélioration des méthodes logiques est la prise en compte de l'abondance relative des espèces, connu pour impacter le rôle dans un écosystème \maxime{ref dans micom je crois}. Avec l'utilisation du solveur linéaire clingoLP, où les nombres flottants peuvent être pris en compte, avec l'analyse en flux dans le modèle ASP, la prise en compte d'une abondance relative pourrait être faite. \\
%Du point de vue du calcul du score de coopération et de compétition implémenté dans le chapitre \ref{ccmc}, ces derniers utilisent le vocabulaire généré par le solveur logique pour calculer les indices à l'aide du langage de programmation Python. Afin de faciliter l'intégration et son déploiement sur différentes architectures logicielles, trouver un moyen de le calculer directement en ASP serait une plus value. \\
%Enfin, dans le chapitre \ref{enrichissement}, une proposition d'intégration temporelle au sein du formalisme logique a été faite. Nous sommes partis du postulat que le formalisme Telingo, s'inspirant du formalisme logique utilisé dans le chapitre \ref{ccmc}, puisse passer à l'échelle. On remarque tout d'abord que sans aucunes contraintes \textit{a priori}, le nombre de modèles possible explose. Ainsi, une première façon de procéder serait d'ajouter des règles logiques à partir de cas concrets d'observations afin d'éviter d'observer des faux-positifs, et ainsi diminuer les sorties de modèle. Une autre amélioration possible concerne l'intégration du temps. En effet, le temps est représenté comme une séquence d'action ordonnée discrète et non comme une dimension dynamique continue. Un premier travail consisterait à implémenter cette dynamique à l'échelle d'un organisme pour suivre la production ou la consommation de métabolites.


%\section{Approches hybrides découplées et couplées}
%L'approche de modélisation hybride développée dans cette thèse à été développée sous de multiples angles. La première est la création d'une approche hybride découplé, avec les projets TANGO et CoCoMiCo. Chacun des deux permet de caractériser une communauté bactérienne en mettant en avant les interactions entre bactéries. TANGO et CoCoMiCo apportent une information supplémentaire et complémentaire sur la communauté bactérienne. Ces deux approches sont complémentaires du fait du raisonnement utilisé dans le modèle logique, le criblage permet de mettre en avant l'ensemble de interactions métaboliques possibles en fonction de l'environnement nutritionnel et peut ainsi servir de support pour inférer les éléments à suivre dynamiquement dans le modèle dynamique (modèle hybride couplé). La modèle dynamique quand à lui, permet à partir d'une base de fait établie, issu de la connaissance biologique, de la littérature ou bien d'un logiciel de criblage, de décrire avec précision des mécanismes de régulation et entre autre, des interactions bactériennes.\\

%En plus de la fiabilité des éléments criblés à vérifier, réduire leur nombre ainsi que obtenir un population microbienne minimale est nécessaire afin d'utiliser des systèmes dynamiques tels que celui de TANGO. En effet, plus le nombre de paramètres à suivre augmente, plus le temps de calcul croit. Plusieurs questions surviennent alors quant aux critères de sélection permettant de mettre en place un filtre efficace. Dans \citep{Frioux2018}, les auteurs proposent de sélectionner une communauté bactérienne basée en minimisant la taille de la communauté. En fonctionnant ainsi, minimiser la taille ne permet pas de garantir que la communauté finale est la meilleure pour répondre à un objectif. En effet, il permet de résoudre la combinatoire du nombre de communauté bactérienne que le programme résout.

%\ifdefined\FromMain %
%\else %
 \end{document} %



